\documentclass[a4paper, 12pt, twoside, openright]{article}
\usepackage{t1enc}
\usepackage[utf8]{inputenc}
\usepackage{graphicx}
\usepackage{tikzsymbols} %ikonokhoz
\usepackage{hyperref}
\hypersetup{
	colorlinks,
    citecolor=black,
    filecolor=black,
    linkcolor=black,
    urlcolor=black
}
%hivatkozások feketék legyenek
\usepackage{url}
\def\UrlBreaks{\do\/\do-}
%megtörheti az urleket a / és a - karaktereknél
\usepackage{sidecap}
\usepackage{wrapfig}
\usepackage{upgreek}
%\usepackage{glossaries}
\sloppy
\frenchspacing
\usepackage[inner=1.5cm,outer=1.5cm]{geometry}
%\geometry{margin=2.5cm}

\usepackage{textcomp} %~ hullám miatt
\usepackage{float} %ábrákhoz
\usepackage{pdfpages}
\newcommand{\tab}{\hspace*{1em}}
\renewcommand*{\thefootnote}{(\arabic{footnote})}

\begin{document}

\title{Ifis játékok}
\maketitle

\begin{itemize}
\item \textbf{Hurkapálcás:} Párokban játsszuk. Minden párnak kell két hurkapálca. Egymással szemben állnak. Mutatóujjaikat felfelé tartják és a legelső ujjperceik közé beszorítanak 1-1 hurkapálcát. Nem a saját két kezük közé, hanem az egyik ember jobb kezének mutatóujja és a másik bal kezének mutatóujja tart egy hurkapálcát. A játék célja, hogy ne essen le vagy törjön el egyik hurkapálca sem. Ha minden pár felállt, akkor indulhat a játék, akinek akár az egyik hurkapálca leesik, az a pár kiesik. nem lehet a hurkapálcát megfogni, csak mutatóujjal lehet tartani. A játék indulása után a többi párt lökdösni kell, hogy elejtsék a hurkapálcát. Csípővel, vállal, könyökkel lehet lökdösni a másikat vagy annak hurkapálcáját, de fokozottan figyelni kell, hogy vigyázzunk egymásra. Az utolsó pár nyer.

\item \textbf{Családos:} Létszámtól függ, hogy egy család 3 (apa, anya, fiú) vagy 4 tagú (apa, anya, fiú, lány). Papírokat előre el kell készíteni, amire a család vezetékneve és a családtag szerepe van írva. (pl.: \emph{Takács apa} vagy \emph{Szakács lány}). Hasonló vezetékneveket kell kitalálni (Takács, Makács, Bakács, Szakács, Lakás, ...). A játék előtt kiteszünk egy nagyobb körbe annyi széket ahány család van és mindenki kap egy papírt, amit még nem nézhet meg. Ahogy indul a játék meg lehet nézni a nevet és a cél, hogy egy széken egy család legyen úgy, hogy az apa ül a széken a térdén az anya, aztán a fiú (majd a lány). Ezt úgy érik el, hogy mindenki üvöltözni kezd, hogy ki hiányzik az ő családjából (az apa célszerűen már le is ül és úgy kiabál). Amelyik család utolsónak ül le annak minden tagja kiesik. A többiek mind odaadják a papírjukat a játékvezetőnek, aki újra kiosztja azokat és kezdődhet a következő kör. Az utolsó család tagjai nyernek.
(Értelemszerűen minden körben új családhoz tartozik egy ember. Vagy 3 vagy 4 tagú családokkal játsszuk, egy játékon belül ne keverjük.)

\item \textbf{Kő, papír, olló két csoportban:} Két csoportra osztjuk a csapatot és a pályát keresztbe kettéválasztjuk egy -- képzeletbeli -- vonallal. Legalább akkora pálya kell, hogy a csapatok mögött legyen 4-5 méter a pálya végéig. Mindekettő cspat megbeszél maguk között egyet a kő-papír-olló kézmozdulatai közül, majd felsorakoznak a vonal két oldalára egymással szemben. A játékvezető jelére elhangzik a ``kő-papír-olló'' és mindkettő cspat mutatja a saját jelét. Amelyik csapat jele üti a másikét annak a csapatnak meg kell fogni a másik csapat tagjait, mielőtt azok visszafutnának a saját alapvonalukig. Akit elkapnak előtt az átáll a másik csapatba és új kör kezdődik. Aki visszajut az alapvonalig, az marad azon az oldalon a következő körig. Ha valamelyik oldal elfogy, akkor a másik csapat nyer.

\item \textbf{Ülj le az üres székre, a többi akadályozza meg:} A teremben össze-vissza szét vannak szórva székek, annyi ahányan játszanak. Egy ember nem ül széne, a többiek igen. Az álló embernek a célja, hogy leüljön az üres székre, de nem futhat. A többiek ezt úgy tudják megakadályozni, hogy mielőtt leülne a saját helyükről felállva átülnek az üres székre. Ilyenkor az álló ember célja az újonnan üreséé vált székre való leülés. Akik ülnek, azok futhatnak a székek között. Ha valaki felállt egy székről, akkor nem ülhet vissza ugyanoda. Ha az álló embernek sikerül leülnie, akkor aki helyére leült, az lesz a következő álló ember.

\item \textbf{Pénzfeldobós, kézszorítós:} Két csoportra osztjuk az embereket és van egy játékvezető. A csoportok beállnak egymással párhuzamosan két sorba, úgy hogy minden ember háttal áll a másik csapatnak. A sor egyik végén van a játékvezető és egy pénzérme a kezében a másik végén egy széken egy könnyen elvehető tárgy, például egy műanyag pohár. A pohárnak ugyanolyan távol kell lennie a két utolsó embertől. A sorban álló emberek megfogják egymás kezét. A játékvezető feldobja a pénzérmét és a kezére csapja, majd megmutatja a két első játékosnak a sorban. Ha fej van a pénzérmén, akkor mindkettő első játékos megszorítja a saját csapatában álló második játékos kezét, aki továbbszorít és így végigmegy a szorítás mindkét soron. Amikor az utolsó emberhez ér a szorítás, az felkapja a poharat. Amelyik csapat hamarabb fogta meg a poharat az kapja a pontot és a poharat megfogó játékos előrejöhet a sor elejére. Ha valami miatt tévesen kapta fel a játékos a poharat (pl. valaki tévesen szorított), akkor a tévesztő csapatból az első játékos megy a sor végére. A játék célja, hogy az egész csapat körbeforogjon (a jó irányba) és az első játékosuk újra az első helyen legyen. A játékvezető minden pénzfeldobás után várjon pár másodpercet mire újat dob, így nem egyértelmű, hogy milyet dobott előbb. Minden játékosnak kifelé kell néznie, legfőképpen nem szabad a másik csapatot és azok kezét nézni. A pénzérme jól látható legyen mindkettő első ember számára. Amikor cserélődik az első ember, célszerű megmutatni, hogy a pénzérmének melyik fele a fej.

\item \textbf{Kapd el a párod:} Két ugyanakkora létszámú csoport. Mindkettő csoport tagjai beszámozva egytől a csoportlétszámig. Kettő vonalat kell húznunk (min 5-6 méterre egymástól), amin kívül feláll két oldalt a két csapat. A pálya közepére egy tárgyat kell tenni (pl. palack, babzsák). A játékvezető mond egy számot, mindkettő csapatból az adott számú játékos célja, hogy vagy megszerezze a tárgyat és visszafusson vele a saját alapvonalára vagy megfogja a másik játékost, miközben annál van a tárgy. Ahogy a szám elhangzott befuthat a két játékos. Nem érhetnek egymáshoz csak ha valamelyik már megérintette a tárgyat. Amelyik játékos hozzárt a tárgyhoz, az olyan mintha a kezében lenne, tehát ha csak egy ujjal is hozáér, a másik játékos már megfoghatja. Ha megfogja, akkor a fogóé a pont; ha viszont a tárgyat megfogónak sikerül visszafutnia az alapvonalához, mielőtt a másik játékos megérintené, akkor az ővé a pont. A játék elején tisztázni kell, hogy teljes testtel vissza kell érni az alapvonal mögé vagy elég, ha valamelyik testrész már éri a vonalat (helytől függ, hogy melyik a jobb megoldás).

\item \textbf{Kézcsapkodós:} Körben leülnek az emberek egy asztalhoz és mindenki beteszi a kezét az asztalra, úgy hogy a mellette levővel keresztezze a kezét, tehát az én bal kezemtől jobbra van a tőlem balra ülő jobb keze, aztán attól jobbra a tőlem jobbra ülő bal keze, majd utána jobbra az én jobb kezem. Így formálunk egy kört. A játék során egy ``jel'' megy körbe. Valaki elkezdi egy tapsolásal az asztalon, a tőle jobbra levő kéz a következő, szintén egy csapással és így megy körbe a csapás. Ha valamelyik soron következő kéz duplát casp, akkor megfordul az irány és az előtte csapott kéz jön. Ha valaki rosszkor csap, annak a rosszul csapó keze kiesik. Rossz csapás, ha nem az a kéz következik; ha több mint 2 másodpercig vár vagy ha nem egyet vagy kettő csap. A csapásoknak gyorsaknak, hallhatóaknak kell lennie és egyértelműen meg kell tudni különböztetni az egy csapást a kettőtől. Aki felemeli a tenyerét az már csapásnak a számít. Minden nyugalomban levő tenyér (tehát, aki épp nem csap), az asztalon kell, hogy legyen. Amelyik két kéz marad utoljára az nyert. Ha valakinek a két keze még játszik, de mindkettő kéz kiesett közüle, akkor ugyanúgy megy tovább a játék, neki a két keze egymás mellé került.

\item \textbf{Üsd a jobboldalit:} Körben ül a társaság, eggyel több szék van, mint ahány játékos. A játék indulásakor beszámozzuk az embereket és mindenkinek az lesz a száma a játék végéig. Az emberek úgy ülnek a széken, hogy ajobb kezük a jobb térdükön van és a bal térdük szabad. Amelyik játékostól jobbra van az üres szék, az hív egy számot. Akinek a számát hívta, annak az a célja, hogy felálljob és átüljön az üres székre a hívó mellé. Aki a hívott számtól balra van (tehát akitől jobbra van, akit hívtak, innen a játék neve), az megpróbálja megakadályozni, úgy hogy mielőtt felállna a hívott játékos a jobb kezével annak a combjára csap. Ha még nem állt fel ez idő alatt a hívott játékos, akkor ott kell maradnia, mert lecsapták. Akármeddig játszható, nincs egy konkrét végkimenet. Ha a hívó játékos véletlenül a saját számát hívja vagy azt aki tőle kettővel jobbra van (azaz az üres széktől eggyel jobbra), akkor aki az üres szék túloldalán van, megcsaphatja büntetésből.

\item \textbf{Szereted a szomszédodat?:} Körben ül a társasá, eggyel kevesebb szék van, mint ahány ember és egy ember áll a kör közepén. A középső ember odamegy egy játékoshoz és megkérdezi tőle, hogy \emph{``Szereted a szomszédodat?''}, mire az igennel vagy nemmel válaszolhat. Ha nemmel válaszol, akkor mindenki eggyel jobbra ül. Ha igennel válaszol, akkor a középen álló megkérdezi, hogy \emph{``És miért?''}, mire az mond egy állítást, például \emph{``Mert kék a szeme.''}. Ekkor mindenkinek akire igaz az állítás (tehát kék a szeme), fel kell hogy álljon és keresnie kell magának egy másik üres helyet. A középen állónak az a célja, hogy leüljön egy üres helyre (akkor is ez a célja, amikor nemmel válaszolt valaki, csak akkor nehezebb leülni). Aki bent marad a helykeresés után, az lesz a következő kérdező. Aki a tulajdonság miatt feláll és átül máshova az nem ülhet vissza a saját helyére (ha nincs más hely, akkor ő marad középen) és amennyiben az üres helyek engedik, nem is ülhet a korábbi helye mellé jobbra vagy balra. A mikor valaki igennel válsazol és mond egy tulajdonságot, annak nem kell igaznak lennie a szomszédjára, azaz a mellette ülőre.

\item \textbf{Asztalon pingponglabdafújás:} Négyzet alakú asztalt kell szerezni, aminek legalább egy méter szélesek az oldalai, de inkább 1.5 vagy több (jatékosszámtól függ). Lehet két asztalt is egymás mellé rakni, csak ne legyen köztük szintkülönbség. Két csapatot alkotunk és mindegyik csapat még ketté osztja magát, de ők együtt vannak. A két csapat elhelyezkedik az asztal négy oldalán úgy, hogy csapattársak egymással szemben legyenek (déli és északi oldalon az egyes csapat, keleti és nyugati oldalon a kettes csapat). Célszarű annyian játszani, hogy mindegyik oldalra jusson legalább 2 ember (min 8 fő; sok ember esetén lehet több asztalon párhuzamosan játszani). A játék kezdetén leteszünk az asztal közepére egy pingponglabdát és egy jelre mindenki elkezdheti fújni. Ha valamelyik csapat oldalvonalán leesik a labda, akkor a másik csapat kap pontot (előző példában ha a déli vagy északi oldalon esik le, az az egyes csapat hibája, tehát a kettes kap pontot). Az asztalhoz kézzel hozzáérni nem szabad maximum vállal. Az asztal fölé behajolni csak annyira lehet, hogy az állunk éépen az asztal fölött legyen. Semmilyen eszközzel nem lehet hozzáérni a labdához, azt csak fújni lehet, ha valaki mégis hozzáér (kézzel, arccal, ...), az olyan mintha leesett volna az ő vonalán a labda. Adott ideig megy a játék, amelyik csapat több pontot szerez az nyer.

\item \textbf{Evolúció:} A játék során mindenki tojásként indul és az a cél, hogy superman legyen. A lépések: tojás $\to$ csirke $\to$ sas $\to$ griffmadár $\to$ superman. (Szükség esetén kiegészíthető több lépéssel.) A tojásnak guggolva kell járnia összekuporodva. A csirkének szintén guggolva, de kissé felemelkedve; az ökleit a mellkasához szorítja és a két karjával csapkod. A sas felegyenesedve kinyújtott karral csapdos. A griffmadár hátul összefogja a két kinyújtott kezét és azt megemeli amennyire tudja, kissé előre is dőlhet. A superman egyik ökölbe szorított kezét kinyújtja (mint Superman \Smiley). A játék során minden játékos keres egy ugyanolyan szinten levő játékost és megküzd vele. Aki nyer az továbbfejlődik, aki veszít az visszalép egyet a fejlődésben (ha tojás volt, akkor tojás marad). A megküzdés legegyszerűbb módja a \emph{kő-papír-olló}, de mást is lehet választani. Például: \\
-- Mindenki kap három papírgalacsint. Amikor ketten párbajoznak, mindegyikőjök az egyik markába vesz a három galacsinjából valamennyit és előreteszi ezt a kezét. Aki kezd, mond egy tippet, hogy szerinte kettejüknek közösen mennyi galacsin van a kezükben (0-tól 6-ig), ezután a másik játékos is mond egy számot, ami nem lehet az mint a másiké. Ezután felfedik kinél mennyi van és ha valamelyikőjük eltalálta, akkor az nyert; ha egyikőjük sem, akkor párbajozhatnak újra vagy kereshetnek másik párt.
-- Ujjpárbaj. Jobb kezüket összekulcsolják, a hüvelykujjuk felfelé mutat. A cél, hogy ahüvelkyujammal leszorítsam a másik hüvelykujját 3 másodpercig.

\item \textbf{Sötétes gyilkosos tapsolással:} A játékvezető kijelöl egy gyilkost a társaságból, úgy hogy a többiek ne tudják ki az. (Körbeállítja az embereket, mindenki a kör közepe felé fordul és becsukja a szemét. Körbejár a játékvezető és megérinti valakinek a hátát.) A játék indulásakor lekapcsoljuk a villanyt és sötétben játszunk. Az emberek járkálnak a teremben és ha összeütköznek egymással, akkor tapsolniuk kell egyet-egyet. A gyilkos úgy tud gyilkolni, hogy ő nem tapsol (megteheti azt is, hogy tapsol, de gyilkolni a nem tapsolással tud). Akit megölt a gyilkos, annak várnia kell 6-8 másodpercet, aztán hatalmas sikoltások között összeesnie. Ha valaki a játékosok közül úgy gondolja, hogy tudja ki a gyilkos, akkor odamegy a játékvezetőhöz és elmondja a tippjét. Ha kitalálta ki a gyilkos, akkor vesztett a gyilkos. Ha nem találta ki, akkor meghal a tippelő is meg az is, akire tippelt. Ha a játék végén már csak a gyilkos marad, akkor ő nyert.

\item \textbf{Kacsintós gyilkos:} (Hasonló az előzőhöz.) A társaságban kijelölünk egy vagy két gyilkost. A játék során, mindenki össze-vissza sétál a teremben; akire rákacsint egy gyilkos az kiesett. Csak a gyilkosok kacsinthatnak. Közben lehet jönni a játékvezetőhöz tippelni, hogy ki a gyilkos; ha eltalálta, akkor a gyilkos kiesik (egy gyilkos esetén vége a játéknak); ha nem találta el, akkor a tippelő esik ki és az is, akire tippelt. Ha a gyilkos egy másikra kacsint az nem esik ki. Aki meghalt az a kacsintástól várjon legalább 6-8 másodpercet és csak utána jelezze hangos jajveszékeléssel, hogy meghalt. Ilyenkor feküdjön le vagy üljön le egy székre a terem szélén.

\item \textbf{Gyilkosos:} \url{https://hu.wikipedia.org/wiki/Gyilkosos}

\item \textbf{Székfoglaló:} Körbe rakunk eggyel kevesebb széket, mint ahány játékos van kifelé fordítva. A játék alatt megy a zene (kell valaki, aki ezt kezeli) és a játékosok körbe-körbe sétálnak a székek körül. Amikor elhallgat a zene, mindenki megpróbál leülni egy székre; abból viszont eggyel kevesebb van, mint ahány játékos, ezért akinek nem maradt szék, az kiesik. A kieső játékossal együtt egy széket is ki kell venni és új kör kezdődik. A játékosok elfogynak és aki a végére marad az nyer. Fontos (főleg a végén), hogy aki a zenét kezeli, az ne láthassa a játékteret és így ne tudjon kedvezni egyik játékosnak sem.

\item \textbf{Vedd ki a szék alól a tárgyat:} Körben ülnek a játékosok, egy szék a kör közepén van, ami alatt van egy tárgy, amit meg kell szerezni. A középső székre egy önkéntes ül, akinek be kell kötni a szemét. A játékvezető kijelöl egy játékost, aki megpróbálja elvenni a tárgyat. Csak csöndben lehet játszani. A közeépen ülőnek 3 lehetősége van eltalálni, hogy merről jön a rabló; ezt úgy jelzi, hogy kinyújtott karjával megjelöl egy irányt. Ha a rabló ebbe a félegyenesbe esik, akkor sikerült a középsőnek megóvnia a tárgyat és ilyenkor helyet cserél a rablóval. Ha a rabló el tudja venni a tárgyat úgy, hogy a középső nem találja ki merről jött (vagy a középen ülőnek elfogyott a három tippje), akkor nyert, ilyenkor a következő körben is marad ugyanaz középen. Ha van rá lehetőség lehet kis csokival játszani és aki megszerez egyet az meg is eheti.

\item \textbf{Halk helycsere:} Körben ülünk, és egy embernek bekötjük a szemét, akit beküldünk középre. Mi a két ember aki hallja a számát cseréljen henden embert beszámozunk (még a középsőt is). A középen álló mondd két számot és akié a két szám, nekik helyet kell cserélniük úgy, hogy a középen álló ne tudja őket megfogni. Elég nagy kör kell hozzá, hogy a középső szabadon tudjon mozogni és figyelni kell, hogy ne okozzon sérülést a körben ülőknek, miközben próbálja megfogni a két embert. Ha valamelyiket sikerül megfognia, akkor azzal helyet cserél és új kör kezdődik. Ha nem sikerül megfognia egyikőjüket sem, akkor a társaság hangosan jelzi a középen állónak, hogy sikerült a helycsere; ilyenkor új számpárost hív. Célszerű minden hívás előtt az ülő embereknek új helyet keresni, hogy ne tudja a középső hányas szám merre van.

\item \textbf{Mondj egy nevet mielőtt fejbe csapnak:} Körben ül a társaság és mindenkinek van egy kitalált neve (ha kevésbé ismerik egymást, akkor lehet mindenkinek a saját és jó névgyakorlásnak) egy adott témakörhöz kapcsolódva (pl mesehős, zöldség, márka, \dots)

\item \textbf{Mániákus család:} Egy embert kiküldünk a teremből, addig a megbeszélünk egy mániát, ami az egész társaságra jellemző és nem túl rejtett. Például, hogy megvakarjuk valamintket vagy a lábunkat keresztbe tesszük, amikor beszélünk. Visszahívjuk a kiküldött embert és ő elkezd kérdéseket feltenni bizonyos embereknek. Amikor azok válaszolnak a mániájukat kell végezni közben. Minden kérdésnél van egy lehetősége a kérdezőnek, hogy kitalálja, hogy mi a mánia, ha sikeül neki, akkor másik játékost küldünk ki és másik szokást találunk a családnak.

\item \textbf{Portás:} Egy embert kiküldünk a teremből, ő lesz a portás; addig a megbeszélünk egy jelenetet, ami a szállodai szobában gondot vagy valamilyen igényt okoz (pl egér van a minibárban). Amikor visszajön a portás kezdődhet a jelenet. Egy (vagy két) választott ember előadja, hogy mi a panaszuk vagy igényük, de a vendég(ek) süketek és némák, ezért csak mutogatva tudnak kommunikálni a portással. Ha aportás úgy érzi, hogy kitalálta, akkor elmondja hangosan és ha igaza volt, akkor lehet új jelenetet faragni ha nem, akkor folytatódik, amíg ki nem találja. Lehet más helyszín is, nem kötelező a portás és a szálloda.

\item \textbf{A cipőt fogd:} Énekes játék. A játékosok körben vannak guggolva. A jobb kezükben a saját egyik cipőjüket tartják. A cipőt jobbra adják egymás elé egyszerre ezzel az énekkel:\\
``A cipőt fogd, hamar ide tedd elém, ugye
szól ki tudja, hol van az enyém?''

A dal ritmusára kell a cipőt továbbadni. A \emph{``fogd''}, \emph{``tedd''}, \emph{``szól''} szavakra tesszük jobbra a cipőt és engedjük el. Az utolsó 3 cipőrakás:\\
\emph{``hol van''}: jobbra a cipőt, de nem engedem el\\
\emph{``az e-''}: balra a cipőt, de nem engedem el\\
\emph{``-nyém''}: jobbra a cipőt és elengedem.\\

Közben a ritmus egyre gyorsul. Aki előtt feltorlódik a cipőhalmaz, az kiesik a játékból.

\item \textbf{Activity bibliai szavakkal:} Az activity játék, csak bibliai szavakkal. 

\item \textbf{Juttasd a kaját a szádba a homlokodról:} Minden játékos kap egy kekszet, amit feltesz a homlokára és onnan kell a szájába juttatnia úgy, hogy az nem érhet az arcán kívül máshoz. Tehát az arcizmait használja ehhez. Ha leesik, akkor a homlokáról kell újrakezdenie. Aki a leggyorsabban megcsinálja az nyer vagy aki adott időn belül a legtöbbet a szájába juttatja.

\item \textbf{Malacos:}A játék során minden ember a saját papírjára rajzol egy malacot, attól függően, hogy hányast dobott egy dobókockával. A társaságot 4-6 fős csoportokra osztjuk és a csoportok leülnek egy-egy asztal köré. Minden játékos kap egy papírt és egy tollat, valamint minden asztal egy dobókockát. A játék indulásakor minden csoport egyik tagja dob a dobókockával majd továbbadja az asztalon belül és így megy körbe a kocka. Ha valakinek sikerül lerajzolnia a teljes malacot, akkor elkiáltja magát, hogy ``malac'' és minden csoport megáll. A malacrajzolás szabályai. \\
1 - test (1 kell belőle)\\
2 - fej (1 kell belőle, csak akkor rajzolható, ha már van test)\\
3 - szem (2 kell belőle, csak akkor rajzolható, ha már van fej)\\
4 - fül (2 kell belőle, csak akkor rajzolható, ha már van fej)\\
5 - farok (1 kell belőle, csak akkor rajzolható, ha már van test)\\
6 - láb (4 kell belőle, csak akkor rajzolható, ha már van test)\\
Amíg a játékos nem dob 1-est, addig nem tudja elkezdeni a rajzolást. Minden testrészből maximum annyi lehet, ami fentebb fel van tüntetve (tehát nem lehet 3 füle). Egy körben egy malacot kell rajzolni, tehát ha dobtunk egy 1-est, majd körbeért a kocka és megint 1-est dobunk, akkor nem tudunk semmit rajzolni. 6-os után újra lehet dobni, akkor is ha nem tudunk lábat rajzolni. Egy példamalac:\\
\includegraphics[scale=0.65]{malac.png}\\
Ha vége egy körnek, mert valaki ``malac''-ot kiáltott, akkor mindenki megszámolja, hogy az ő malaca mennyit ér. Minden testrész annyit ér, amennyit kellett dobni a lerajzolásához. A teljes malac 46 pont ($1+2+2*3+2*4+1*5+4*6$). A játékosok felírják a malacuk mellé a pontot és minden csoportból az adott körben legtöbb és legkevesebb pontot szerző játékosok felállnak és csoportot cserélnek, a legtöbbet szerző játékosok egy asztallal jobbra mennek a legkevesebbet szerzők egy asztallal balra mennek. Ha valamelyik asztalnál több legtöbb vagy legkevesebb pontot szerző játékos is van, akkor kő-papír-ollóval döntsék el, hogy ki marad és ki megy az asztaltól. Jöhet a következő kör. Akármennyi kört lehet játszani. Arra figyeljünk, hogy az első malac ne foglalja el az egész lapot. A végén össze lehet adni az összes pontot és egy végső nyertest hirdetni.

\item \textbf{Ninja:} A játék során az emberek megpróbálják a kezükkel lecsapni a másik játékos kézfejét és ezzel kiejteni, így a végén csak egy ember marad. A játék elején mindneki benyújtja középre a kezét, majd a játékvezető szavára mindenki egy-két nagy lépéssel hátralép és mozdulatlan marad. Ahogyan a benyújtott kezek közben sorban voltak a játékosok, úgy megyünk sorba. A kezdő játékos megpróbálja egy mozdulattal lecsapni egy bármelyik másik játékos kézfejét. Csak kézfejjel lehet lecsapni és csak kézfejre csapás ér (karórát, gyűrűt, \dots{} célszerű előtte levenni). A csapásnak egy mozdulatból kell állnia, azaz közben léphet előre/hátra, de célirányosan a másik kézfeje felé kell csapnia, ha támad. Lehet védekezőmozdulatot is csinálni (például ha az égbe emeljük mindkettő kezünket), de ennek is egy mozdulatnak kell lennie. Ha valamelyik játékos kézfejére próbál csapni a soros, akkor ő is tehet egy védekező mozdulatot, azaz elhúzhatja a kezét, de ennek is egy mozdulatnak kell lennie. A támadó játékos használhatja mindkettő kezét és így akár a másik játékos mindkettő kezét megcélozhatja. Ha talált, akkor az a játékos kiesik, akinek lecsapták a kezét. Ha nem talált, akkor ott kell hagynia a kezét, ahol lecsapta volna a másikét (tehát az szabálytalan, ha egy nagy karlendítéssel próbálom lecsapni a másik kezét és így irányzom, hogy a saját hónom alá kerüljön vissza a kezem, ha ő közben elhúzza az övét). Ezután a soron következő játékos jön; ez nem biztos, hogy ugyanaz, akit a mostani játékosunk támadott.

\item \textbf{Csoport számoljon el 20-ig:} Több (nagyjából) egyenlő létszámú csapat játssza, célszerűen legalább 5-6 fős egy csapat, de lehet akár 8-10 is. A játék célja, hogy a csapat elszámoljon 20-ig, az alábbi szabályokat betartva. Ha elrontják, akkor elölről kell kezdeniük.\\
Senki nem beszélhet vagy adhat ki hangot, csak ha számot mond. Egy számot csak egyszer lehet mondani. Mindig a soron következő számot kell mondani. Ha valaki mond egy számot, akkor a következőt nem mondhatja a mellette levő és ő maga sem. Nem lehet olyan mozdulatot tenni, vagy kommunikálni, ami a játék kimenetét befolyásolja (nem lehet mutogatni; nem lehet erősen valakire nézni, jelezve, hogy most ő jön; nem lehet kopogni vagy a másikhoz érni).\\
nincs meghatározva, hogy ki kezdi a játékot. A játékvezetőnek célszerű körbejárni a csapatok között, és ha hibát lát, akkor szólni, hogy kezdjék újra.

\item \textbf{Hagyd ki a hármast:} Körben kell lennie a játékosoknak és kijelölünk egy embert, aki kezdi a játékot, utána körben megyünk. 1-től kell mondani a számokat úgy, hogy ami hárommal osztható vagy amiben van 3-as, azt ki kell hagyni. (3,6,9,12,13,15,18,21,23,24,27,30,31,32, ... ) Akinek kéne mondani azt a számot, amit szabály szerint nem lehet kimondani, annak hallgatnia kell és az utána következőnek kell mondania. (tehát a harmadik embernek nem 4-et kell mondania, hanem hallgat és a negyedik ember mondja a 4-et). Ha valaki elrontja, az kiesik. Ha több egymást követő szám van, amit ki kell hagyni, akkor több ember marad ki.

\item \textbf{Kikötözött ceruzával rajzolj valamit:} 3-6 fős csapatokban játsszuk. Előkészületnél egy-egy filctollra/ceruzára kötözünk cérnát vagy spárgát, úgy hogy ki lehessen függőlegesen feszíteni, ha több ember fogja a zsineg végét. Társaságtól függően 20 cm-től 1 m-ig is lehet a zsineg hossza. Olyan íróeszközt kell választani, ami már akkor is fog, ha csak a saját súlya nyomja le. A játék elején meg kell határozni, hogy mit kell lerajzolni a csapatnak (lehet egy kivetített kép és le kell másolni, de lehet csak annyit mondani, hogy rajzoljanak egy autót). Az íróeszközhöz célszerű két pontot, alul és felül legalább 3-3 zsineget rögzíteni.

\item \textbf{Rúgd fel a cipőt az asztalra:} több csapatban is játszható. Egy asztalt helyezünk a játékosoktól bizonyos távolságba (2-6 méter) és a cipőjeiket/papucsaikat fel kell rúgni az asztalra. Leveszik a lábukról és a lábfejük végét visszadugják annyira, hogy tudjanak lendíteni egyet rajta. Előre tisztázni kell, hogy számít-e ha már fent volt a cipő, de egy másik cipő lelökte. Adott idő alatt kell valamennyit felrúgni vagy adot mennyiséget kell.

\item \textbf{Hettyem-pitty:} Körben ülnek a játkosok. Egy embert beállítunk középre és bekötjük a szemét. (Ha olyan a társaság, akkor adjunk neki egy párnát is.) A bekötött szemű odamegy valakihez és beül az ölébe vagy térdére (ha kell, akkor a párnát tegye maga alá), majd azt mondja, hogy ``hettyem'', mire az akinek az ölébe ült, azt válaszolja, hogy ``pitty''. Ekkor a bekötött szemű tippelhet, hogy ki az, ha kitalálja, akkor helyet cserélnek. Ha nem, akkor próbálkozhat mégegyszer. Ha ekkor sem találja ki, akkor tovább kell mennie más emberhez. Lehet az újrapróbálást 3-ra vagy 4-re növelni, ha nagy a társaság vagy nehezen megy a kitalálás. Célszerű eltorzítani a hangunkat. 

\item \textbf{2 igaz, 1 hamis:} Mindenki mond magáról 2 igaz és egy hamis állítást és a többieknek ki kell találnia, hogy melyik a hamis. (Csapatépítő játék, nem feltétlen kell pontozni.)

\item \textbf{Mássz át az asztal alatt:} Egy asztalt felállítunk és a tetejéről indulva át kell mászni alatta és visszaérkezni a tetejére úgy, hogy közben semmivel nem érinthetjük a földet. Pár embernek le kell fogni az asztalt, hogy az ne boruljon fel.

\item \textbf{Bogozd ki a kézcsomót:} Körben állnak az emberek és becsukják a szemüket. A játékvezető szavára mindenki elindul befelé a két kezét a kör közepe felé nyújtva (mint a filmekben a zombik). Ha eléggé közel értünk egymáshoz, akkor megfogunk egy-egy kezet a jobb és bal kezünkkel. A játékvezető (aki szintén játszhat), ha már megfogott két kezet, akkor figyelje, hogy minden rendben van-e (nincs-e olyan hely, ahol három kéz fogja egymást; vagy segítse ha már csak két kéz keresgél és nem találják  egymást). Ezután kinyitja mindenki a szemét és ki kell bogozni a kézcsomót úgy, hogy nem engedhetjük el egymás kezét közben. (Van olyan, hogy nem lehet teljesen kibogozni, de ez csak a végén derül ki). Egymás alatt, fölött kell átbújni.

\item \textbf{Székkel lóverseny:} A teremben kijelölünk egy indulóvonalat és egy célvonalat vagy annyi tárgyat, ahány játékos van és azt kell megkerülni és visszajutni az indulóvonalra. A játékban minden versenyző (célszerű egyszerre csak 2-4 embernek versenyeznie) feltérdel egy székre, a szék háttámlája kell, hogy a pálya felé nézzen és a támlát kell fogni. A játék indítása után nem szabad lelépni a székről és nem szabad semmi mást megérinteni a széken kívül. A támlát megrántva ugrásszerűen lehet előrehaladni. Könnyen fel lehet borulni és sérülést szerezni, szóval csak óvatosan.

\item \textbf{Egymás nyakába a láb, százlábú:} Több csapat felsorakozik egymással párhuzamosan. Pókjárás pozícióba helyezkednek és a lábaikat az előttük levő nyakába/vállára teszik. A sor elején levő ember lába a földön van, mindenki másnak csak a keze (és induláskor a feneke). A játékvezető szavára el kell indulni és egy előre meghatározott távot megtenni. Ha valakinek leér a feneke az nem baj, de nem szakadhat szét a százlábú.

\item \textbf{Piramis:} 6 embernek kell alkotnia egy piramist úgy, ahogy a pompomlányok csinálják. Négykézláb van mind a 6 ember. Alul 3, majd 2, végül 1. Figyeljünk, hogy az alattunk levőnek ne a gerincére rakjuk a lábunk/térdünk. 

\item \textbf{Ujj-párbaj:} 2 ember párbajozik egymással. Lehet úgy, hogy az egész társaság játszik és amikor lemegy egy kör, akkor a nyertesek versenyeznek és a végén egy marad. A páros jobb kezeikkel kezet fognak, de nem a hivatalos kézfogáshoz hasonlóan, hanem mint a feketék. A mutatóujjukat kinyújtják. Amikor a játékvezető elindítja a játékot, akkor az a céljuk, hogy a mutatóujjukkal megérintsék a másikat. (A jobb alkar nem játszik.) Ha valamelyik játékosnak sikerül, az nyer.

\item \textbf{Húzd rá a másikat a székre:} Körben áll a társaság és megfogják egymás kezét. Középen van egy szék (ha van egy nagyobb könnyű tárgy, például egy üres szennyeskosár vagy kuka, az talán jobb). A játékosoknak az a célja, hogy egymást ráhúzzák a székre. Aki hozzáér a székhez, az kiesik és szűkül a kör. Az utolsó két ember nyer. Ha szétszakad valahol a kör, akkor a szakadás mindkét oldalán álló kiesik. Ha van rá lehetőség, akkor szerezzünk 50-60 cm-es köteleket és kössünk mindkét végére csomót; az így kapott 30-40 cm-es köteleket használjuk. A játékosok ezt fogják így ha szétszakad a kör, akkor aki elengedte a kötelet csak az esik ki és nem a szakadás mindkettő oldala. Figyeljünk rá, hogy egyik változatnál se rángassuk a mellettünk álló kezét vagy a kötelet.

\item \textbf{Babzsák férfi-női név:} Körben áll a társaság és egy babzsákot dobálunk egymás között. Minden dobásnak egyértelműen egy ember irányába kell mennie és a dobó személy a dobás pillanatában mond egy keresztnevet. Ha a név azonos nemű, mint akinek dobták, akkor nem kapja el (elhúzza magát előle és a babzsák a földer esik); ha másik nemű a mondott név, akkor el kell kapnia. Ha hibázik az elkapó (elkapja, de nem kellett volna vagy nem kapja el, de el kellett volna), akkor kiesik. Fontos, hogy amikor a dobó kezét elhagyja a babzsák, akkor már a teljes név ki legyen mondva. Mindenki által ismert neveket mondjunk. A babzsákot alulról lendítve kell dobni és nem a másikhoz hozzávágni.

\item \textbf{LE-FEL:} Körben ül vagy áll a társaság és mindenki a saját térdét nézi. A játékvezető azt mondja, hogy ``FEL'' és ilyenkor az összes játékos felemeli a fejét és valakinek a szemébe néz. Ha két játékos egymás szemébe néz, akkor ők nagy jajveszékelés közben kiesnek. Mindig ``FEL'' után kötelező valakire nézni, nem szabad a plafont vagy bármi mást megcélozni a tekintetünkkel. Az utolsó 1-2 ember nyer. Ha egy menet lemegy és kiesnek, akik egymásra néztek, akkor a játékvezető a ``LE'' szóval jelzi, hogy mindenki nézze a saját térdét és új menet következik.

\item \textbf{Ülj le, ha igaz rád az állítás:} Mindenki áll és felolvas egy állítást a játékvezető, akire igaz ez az állítás, annak le kell ülnie. Az állvamaradottak játszanak tovább. Az utolsónak marad vagy akik utoljára esnek ki, azok nyernek. Ha nagy a társaság, abban az esetben új kör indulhat, akkor is, ha még 5-10 -en is állnak, hogy a többiek addig ne unatkozzanak.
/állításokat sorolunk (pl.: szereti a palacsintát) és ha igaz valakire, akkor az üljön le, aki állva marad, az nyer/

\item \textbf{Ülj jobbra ha igaz rád az állítás:} Körben ülünk, a játékvezető felolvas egy állítást, akire igaz az eggyel jobba ül. Aztán újabb állítást olvas fel, ha igaz rád akkor jobbra ülsz (lehet valakinek az ölébe), de ha ül valaki az öledben akkor nem ülhetsz jobbra, még akkor sem, ha pont ő is feláll és arrébb ül. A játék elején jegyezze meg mindenki, hogy melyik székről indult és aki leghamarabb oda visszaér, az nyer (akkor is, ha ha valakinek az ölébe kerül vissza, aki ott ül).

\item \textbf{Rajzold le a pároddal:} Asztalnál állnak párokban, a pár egyik fele elöl a másik hátul. A hátsó az elötte álló hónalja alatt benyújtja a kezét és ceruzával megpróbálja lerajzolni azt, ami a táblára fel van rajzolva vagy ki lett vetítve.  A hátsó nem nézheti sem a rajzot, sem a kivetített ábrát. Az elöl alló mondja neki, hogy merre húzza a vonalat. Csak egyszerű instrukciókat lehet adni (``Rajzolt egy vízszintes vonalat.'', ``Most balra egy kis kört.'', \dots{}), nem lehet egyértelműen a célábrára utaló utasítást adni (pl.: ``Rajzolt egy házat.'') Ha tehetjük, akkor a hátsónak kössük be a szemét; ebben az esetben akár állhatnak egymás mellett is.

\item \textbf{Húzzuk fel magunkat:} Körben ülünk 4-6-an és benyújtjuk a kezünket középre. Megfogjuk egymás kezét és fel kell húznunk magunkat álló helyzetbe, de a talpunk nem mozoghat.

\item \textbf{Lepedőleeresztős névkitalálós:} Két csoportra osztjuk a társaságot. Kell két ember, aki nem játszik, ők fogják tartani a lepedőt, ami nem lehet átlátszó, de még áttetsző sem. A két csapat a mellmagasságban tartott lepedő két oldalán helyezkedik el és (hang nélkül) kiválasztanak egy-egy játékost, aki a lepedőhöz térdel. A többiek kicsit távolabb vannak, hogy egyértelmű legyen ki a kiválasztott. Az egyik lepedőtartó a játékvezető, aki ad egy jelet mire leeresztik a lepedőt és a két kiválasztott játéko meglátják egymást. Amelyikőjük hamarabb ki tudja mondani a másik nevét, az elrabolja a másikat és az ő csapatuk egy taggal bővül. Ha egyszerre hangzik el a két név, akkor helyet kell cserélniük. Előre tisztázni kell, hogy becenevet elég mondani vagy teljes keresztnév, esetleg teljes név számít csak. A csapat többi tagja nem segíthet.

\item \textbf{Lény:} Kell hozzá egy legalább hátom részre szedhető zseblámpa és egyház, aminek van jópár terme, szobája. Valamint teljes sötétség, tehát ha az ablakon beivilágít a hold az már sok. A játék leején kiválasztjuk, hogy ki lesz a lény. A játékvezető kiküldi az embereket (a lényt is) és elrejti a házban a zseblámpa darabjait; ne legyen túl könnyű helyen, de hozzáférhető/megtalálható legyen. A játékosoknak tudniuk kell, hogy hogy néznek ki a zseblámpa darabjai. A játékvezető ezután lekapcsolja alámpákat és szól a lénynek hogy bejöhet. A lény elrejtőzik a házban. Ezután bejönnek az emberek és elkezdik keresni a zseblámpa darabjait. A lény előbújik és elkezdi megfogdisni az embereket. Akit megfogott, az mozdulatlan kell, hogy maradjon. Ha valaki, akit még nem fogott meg a lény megérinti a mozdulatlan embert, az újra mozoghat (kereshet és ki is szabadíthat másokat). Ha mindenkit megfogott a lény, akkor ő nyert. Az emberek célja, hogy megtalálják a zseblámpa darabjait; összeszereljék azt és rávilágítsanak vele a lényre, ekkor az emberek nyernek.

\item \textbf{Kidobó:} Ki kell jelölni előre egy pályát (létszámtól függően) amin kívül van legalább még egy méter szabad hely. Célszerű a szabadban játszani. Minden játékos a pályán belül áll, a jatékvezető háttal bedobja a labdát (lehet két vagy több labdával is játszani). Dobni, csak a pályán kívülről lehet, tehát ha megfogtam a labdát, akkor ki kell mennem vele együtt a pályán kívülre és onnan dobhatok. (A legrövidebb úton kell kimenni.) Ha dobtam, akkor vissza kell jönni. Ha valakit eltaláltak, az leguggol ott ahol kidobták és próbál újra játékba kerülni. Ennek kétféle módja van, egyik ha elkap egy labdát és kidob valakit (neki nem kell kimennie, hanem ahol guggol, onnan dob). Másik módja, ha hozzáér valakihez, aki játékban van a pályán belül. Ekkor akit megfogott, az guggol le és ő újra szabad. Akinél labda van azt nem lehet megfogni (annak pedig kötelessége a legrövidebb úton és időn belül kimenni a pálya szélére dobni). Ha valaki, aki szabad a pályán belülről dob, akkor le kell guggolnia (ha eltalált valakit, az szabad marad). Ha valaki elkapja a labdát, amit felé dobtak, akkor a dobó leguggol (ha alapból guggolt, akkor úgy marad). A guggoló embereknek nem szabad mozogniuk, a talpuk egy helyben kell, hogy maradjon. Aki kimegy a pályáról, az is le kell, hogy guggoljon. Aki utolsónak szabad marad, az nyer.

\item \textbf{Csipesz:} Egész alkalom alatt megy a játék. Valakire rárakja a játékvezető a csipeszt (a ruhájára vagy a hajába) úgy, hogy az ne vegye észre. Amikor az illető megtalálja, akkor megpróbálja továbbadni úgy, hogy akire csipteti, az ne vegye észre. Ha az átadás pillanatában lebukik, akkor nála marad a csipesz. Akinél az alkalom végén van a csipsz, vesztett, lehet csináltatni vele valamit (például ő mosogat).

\item \textbf{Stop:} Egész alkalom alatt megy a játék. Mindenkinek van egy ``stop""-ja és azt rámondhatja az alkalom alatt bárkire, akinek emiatt mozdulatlanul kell maradnia egy percig. Előre szögezzük le, hogy milyen helyzetben lehet használni a ``stop''-okat, hogy véletlenül se rontsuk el az alkalom többi részét.

\item \textbf{Postás:} Körben állunk, egy valaki középen áll. A körben levők közül valaki a levélfeladó, ő küldi a levelet valakinek a körben. Ezt úgy teszi meg, hogy hangosan kimondja, hogy ki a címzett: ``Küldöm a levelemet Józsinak''. Ezután a tőle jobbra vagy balra álló kezét megszorítja (csak az egyiket). Annak tovább kell szorítania és így eljut a levél a címzetthez. A küldő "elküldtem"-mel kell hogy jelezze azt a pillanatot, amikor a mellette levő kezé szorítja. A címzett "megkaptam"-mal jelzi, ha megkapta. A középen álló megpróbálja kitalálni, hogy hol tart a szorítás; hármat tippelhet; emberre kell mutatni, nem kézre; ha a mutatott ember valamelyik oldalán volt a szorítás, akkor aki szorított, az megy be középre. Amikor elindul a levél, a középen álló a küldő szemébe néz, csak az elküldés pillanata után nézhet másfelé, különben nagyon egyszerű lenne. A játék elején meg kell határozni, hogy legalább mennyivel távolabbi embernek kell küldeni a levelet (legalább 4-5 célszerű egy 15+ társaságban), hogy legyen útja a levélnek.

\item \textbf{Rabszolgás:} A játékvezetőhöz a játék előtt egyesével odamennek az emberek és mondanak egy álnevet (ez lehet fogalom, tárgy, név de mindenki által ismert nem hosszú szó kell, hogy legyen), ami az övék lesz, úgy hogy azt a játékvezetőn kívül más ne hallja. A játékvezető felírja egy papírra, hogy ki milyen álnevet mondott (azt is, hogy ki mondta meg az álnevet is). Ha mindenkinek van álneve, akkor a társaság leül egy körbe és a játékvezető felolvassa az álneveket (csak az álneveket és nem olyan sorrendben, amilyen sorrendben mondták neki). Ha sok játékos van, akkor kétszer is felolvashatja, de a játék folyamán többször nem (kivéve, ha nagyon elakadunk vagy még nem ismeri a társaság a játékot). Ezután kezdődik a játék egy választott játékos tippel. Mond egy nevet a társaságból és egy álnevet, pl.: ``Józsi, szerintem te vagy a labda''. Ha Józsi a labda, akkor a tippelő mögé kell ülnie székestül és a rabszolgája lesz, ezután megint az jön, aki az előbb volt. Ha nem találta el, akkor Józsi jön és ő tippel. Ha olyan embert találok ki, akinek már vannak rabszolgái, akkor tudnom kell, hogy nekik mi volt az álvenük (ez már egyszer ugye kiderült) és akkor őket is megszerzem, ha valaki álnevét nem tudom, akkor az kiszáll a játékból. Ha valakinek vannak rabszolgái, azoknak segíteniük kell a gazdájukat; ötletekkel, pl hogy milyen álnevek voltak még, amiket lehet mondani vagy pl ne azt tippelje, hogy ``Sanyi az áfonya'', mert azt már valaki mondta és nem talált. Aki a végén marad az nyer.

\item \textbf{Találd ki ki van rád írva:} Legalább annyi papírral kell előre készülni ahányan játszunk. Minden papírra a játékvezető előre felír egy híres személynevet vagy karaktert (pl.: Mózes, Jack Sparrow, Micimackó, ...). A játék elején mindenkinek a hátára ragasztunk egy nevet, amit ő nem lát. Amikor elindul a játék minden játékos megpróbálja kitalálni, hogy ki van a hátára írva. Ehhez eldöntendő kérdéseket kell feltenniük egymásnak; tehát odamegy az egyik játékos a másikhoz, megmutatja neki, hogy ki van a saját hátán és feltesz egy kérdést (pl.: Férfi vagyok?; Élő vagyok?; Meseszereplő vagyok?), mire a másik igennel vagy nemmel válaszol. Ekkor a másik is feltesz egy kérdést. Ha mindeketten válaszoltak, akkor mást kell keresniük. Egymás után nem lehet többet kérdezni ugyanattól az embertől, csak akkor lehet újra, ha már nincs olyan, akitől még nem kérdeztem. Aki először kitalálja, hogy ki ő az nyer.

\item \textbf{Hering/szardínia/:} Egy valaki elbújik a házban vagy udvaron, a többi elkezdi keresni, aki megtalálta, mellé bújik, az utolsónak megtaláló veszt és bújik el következőleg vagy ha van önként jelentkező, akkor ő.

\item \textbf{Egy forintos az állra:} Mindenki kap egy egyforintost vagy valamilyen hasonló apró tárgyat, de mindenki ugyanolyat kapjon. Feltesszük az állunkra (vagy a homlokunkra, de itt is legyen egység). Az indítás után lökdösni kell egymást a vállunkkal, csípőnkkel. Akinek leesik az egyforintosa, az kiesett. Miután elindult a játék nem szabad semmivel az egyforintoshoz érni vagy megigazítani. Figyeljünk egymásra, hogy ne okozzunk sérülést.

\item \textbf{Kacsintással rablás:} Körben állunk párosával, a pár egyik tagja a körben beljebb áll, mint a másik, aki mögötte áll és az elöl álló sarkát nézi. Egy ember egyedül van, ő rákacsint valamelyik elöl állóra, aki nem a szomszédja; akire kacsintott, az megpróbál odaszaladni hozzá, de a mögötte álló elkaphatja. Ha elkapta, akkor helyet cserél az elkapó és a szaladni vágyó, aki kacsintott egyedül marad és újra kacsint. Ha nem kapja el, akkor átmegy akire kacsintott, ahhoz aki kacsintott és beáll mögé. A hátul állóknak maguk mögött kell tartaniuk a kezüket. A hátul álló csak úgy kaphatja el az előtte levőt, hogy maximum az egyik lábával léphet el; ha mindkét lábával ellép az elkapáshoz, az nem érvényes és az elöl álló átmegy a kacsintó mögé.

\item \textbf{Komolyak és vidámak:} Alkossunk két csapatot, az egyiknek meg kell nevettetnie a másik minden tagját. Nézzük az időt, hogy meddig bírták, utána szerepcsere, és ezt az időt is lemérjük, aki tovább bírta, az a csapat nyer.

\item \textbf{Japán foci:} Körben állunk, terpeszve tett lábbal úgy, hogy összeérjenek a lábaink, tehát ne legyen rés a lábak között. Egy kisméretű labdát kel ütögetni a körön belül és akinek a lába között átmegy a labda, kiesett; ekkor szűkítjük a kört.A labdát csak ütni lehet, megfogni nem. A szomszédot nem lehet kiejteni (tehát az utolsó 3 ember nyer). A labdának a földön kell maradnia. Lehet azt csinálni, hogy aki térd fölé üti vagy magasan a körön kívülre, az kiesik.

\item \textbf{4 sarok:} Ha van lehetőség, használjunk ehhez a játékhoz kivetítőt. A játékvezető felolvas/kivetít egy kérdést és négy válaszlehetőséget, amiből csak az egyik helyes. A terem négy sarka jelöli a négy válaszlehetőséget (A, B, C, D sarok). Minden játékos beáll abba a sarokba amelyik válasz szerinte a helyes. Ha mindenki elhelyezkedett, a játékvezető elmondja a jó választ. Aki nem abban a sarokban állt, az kiesett. A maradék játékosnak szól a következő kérdés, az utolsónak maradó játékos nyer. Másik változatban, nincs kiesés; mindenki számolja, hogy mennyi helyes válasza volt és a végén a legtöbb helyesen válaszoló nyer.

\item \textbf{Válogasd szét:} 3-5 csapatban játsszuk, 3-8 fős legyen egy csapat. Mindegyik elé kiöntünk egy kupac keveréket (pl.: rizs és búza, vagy kukorica és bab) és szét kell válogatniuk minél előbb és pontosan. Ha valamelyik csapat elkészült, jelzi és a játékvezető ellenőrzi, hogy hibátlan-e. Ha pontos a szétválogatás, akkor nyert a csapat; ha van benne hiba, akkor a következőnek jelző csapaté az esély. Lehet úgy is játszani, hogy mérjük az időt, onnantól, hogy az első caspat jelzi, hogy végeztek és minden hibájukért valamennyi időlevonás jár (pl minden rossz kupacban levő búza/rizs +1 perc). Ha így is a többi csapat előtt végeznek, akkor nyertek. Természetesen a többi csapat kupacait is ellenőrizni kell.

\item \textbf{Kevés kéz és láb:} Megadott számú kéz és láb lehet csak lent a földön (pl. 15 embernél, 7 láb és 5 kéz). Olyan módon kell egymásra mászni, összekapaszkodni, hogy az előírt kéz- és lábszám teljesüljön.

\item \textbf{Bekötött szemű embert vezetni kell:} Kell egy vagy több akadálypálya és a bekötött szemű embert vezetik a többiek. Csak beszélni lehet hozzá, nem lehet hozzáérni. Egyszerre játszhatja több csapat is és nézzük, hogy kinek sikerül előbb visszaérnie. Lehet úgy is játszani, hogy csak egy ember beszélhet egy csapatból. Van olyan változat is, hogy a pálya szélén kell maradnia a beszéddel irányító embernek/csapatnak.

\item \textbf{Crebs soccer:} 4 csapat pókjárásba áll egy négyzet 4 oldala mentén. Mindenkinek van egy száma csapatonként, tehát például mindegyik csapat 1-től 10-ig be van számozva, középen van egy labda. A játékvezető mondd egy számot, akiké a szám, azok bemennek pókjárásba és megpróbálják a labdát kirúgni valamely másik csapat fölött. Amelyik csapat fölött kimegy a labda, az mínusz egy pontot kap. Jöhet a következő kör. Figyelni kell egymásra. Ki is lehet kötni, hogy nem szabad a másikhoz érni.

\item \textbf{Jelelős:} Mindenkinek van egy saját jele, ami egy mozdulat (pl. orrvakarás; erőltetett, széles mosoly; levegőbe rúgás). Egy ember áll középen, valaki (akiről nem tudja a középső, hogy ő az) kezdi a jelküldést, valaki más jelét mutatja és az az ember a saját jelével fogadja, ugyanígy azután akihez ment az küldi tovább; a középső tippelhet időközönként, hogy hol a jel (ne sűrűn, hogy ne legyen lelőve a lényeg). Addig nálam van a jel, amíg a másik el nem fogadta a saját jelével./

\item \textbf{Amerikából jöttem:} A kiválasztott pár közös megbeszélésük alapján valamilyen foglalkozást mutogat el. A többieknek pedig ki kell találni mi is az a foglalkozás. Ha kitalálták, cserélnek. A mutogatók természetesen nem beszélhetnek játék közben. A játék elején, amikor bejön a terembe a pár, azt mondják, hogy ``Amerikából jöttünk, mesterségünk címere: V-Ő''. A mesterség címere a foglalkozás első és utolsó betűje, például villanyszerelő esetén V-Ő. Amikor valki kitalálta, akkor a pár tagjai megkérdezik, hogy ``Kit választasz, cicát vagy kutyát?''. Ezt is előre meg kell beszélniük, hogy a melyikőjük a cica és melyikőjük a kutya. A kitaláló amelyiket választja, azzal megy ki a következő körre. A másik beül a játékosok közé. Lehet cica és kutya helyett akármi más.

\item \textbf{Faroklopás:} Minden ember kap egy kendőt (ugyanolyan kendőkre van szükségünk). Be kell tűrni a nadrágunkba hátulra úgy, hogy a kendő nagy része kilógjon és könnyen ki lehessen húzni (nem lehet megkötni, nem takarhatja a póló). Amikor indul a játék mindenkinek az a célja, hogy a másik kendőjét kihúzza. Ha valakinek kihúzom a kendőjét, akkor az kiesik, a kendőjét ledobom a földre és megyek tovább játszani, az ő feladata, hogy összeszedje, és kimenjen a pályáról. Amint a kendőt kihúzták, már nincs játékban az illető. Nem szabad egymáshoz érni, csak a másik kendőjéhez. Saját kendőhöz sem szabad érni, a játék elején úgy kell a nadrágba tűrni, hogy ne essen ki magától. Lehet csapatokban is játszani, ekkor célszerű ugyxanabba a csapatba tartozóknak ugyanolyan színű kendőt adni.

\item \textbf{Piros lámpa, zöld lámpa:} Egy ember áll a terem túloldalán. A többiek az ittenin. A túloldali háttal van az embereknek. A játék indulásakor mindenki elindul a túloldal felé, egyszer csak a túloldali visszafordul, ekkor mindenkinek mozdulatlanná kell dermednie. Ha valaki mozog az visszamegy a kezdőhelyre, ezt a túloldali dönt el, hogy ki mozdult meg. Aki eléri a túloldalt, az cserél az ottanival. A túloldali többször is hátranéz és mindig visszaküldi a mozgókat. Egy hátraforulás nem tarthat túl sokáig (6-8 mp).

\item \textbf{5 passzos:} Két csapatra osztjuk a társaságot. Szükséges egy gumi- vagy szivacslabda, és két stabil szék, amire fel lehet állni. Kint érdemes játszani vagy nagyobb teremben. A pálya két végébe felállítunk egy-egy széket. Mindegyik csapat választ magának egy embert (elkapó), akit átküld a pálya másik végére és az feláll a székre. A játékban a cél, hogy a saját térfelünk alapvonalától eljuttassuk a labdát a pálya túloldalára az elkapóhoz. Mielőtt a labdát az elkapónk megfoghatná minimum ötöt kell passzolni a csapatunkon belül. Ha a labda földet, a terem oldalát, széket ér, akkor a másik csapatnak át kell adni a labdát. Ha a másik csapat valamelyik tagja elkapja, akkor ők támadnak (nem kell visszavinni az alapvonalra, onnan jönnek, ahol elkapták); de nulláról indul a passzaik száma. 5 passznál lehet többet passzolni, az a minimum. Egyértelműsíti a pontszerzést, ha az elkapónak adunk egy vödröt, amit feltart és abba kell beledobni a labdát. Tisztázni kell előtte, hogy a kipattanó labda pontnak számít-e.

\item \textbf{Vidd a szádban a vizet:} Több csapatban játsszuk. Mindegyiki csapat felsorakozik egy-egy vonalba egymás mellé. A pálya másik végén, mindegyik csapattal szemben van egy üres vödör (ugyanolyan vödrök kellenek). Minden játékos kap egy üres poharat és kell a játéktér mellé vizvételi lehetőség (csap vagy kancsók vízzel vagy segítők hordják kancsóban a vizet). Amikor indul a játék a legelső emberek a csapatokból bevesznek a szájukba annyi vizet amennyit csak tudnak és átszaladnak a pályán, majd beleköpik a vizet a vödörbe, visszaszaladnak és amikor visszaértek, akkor indulhat a következő. A vizet előre a szájába veheti. A vödörbe csak az emberek szájából juthat víz, ha a pohárral futnak, abból nem önthetnek bele vizet, sem más módon nem juttathatnak bele folyadékot; viszont ami a szájukon kijön az mind számít \Smiley. Amikor lefújjuk a játékot, az a csapat nyer amelyik vödrében a legtöbb víz van.

\item \textbf{Középen egy ember találja ki a dalokat:} Egy ember beáll a kör közepére, amennyire pontosan csak lehet. Egy jelre mindenki elkezd énekelni valamit (hangosan), de mindenki mást. A középen állónak meghatározott időn belül minél többről meg kell mondania, hogy mi a dal címe.


\item \textbf{Jobbkézrecsapós,:} Körben állunk, egy ember körbe jár a körön belül (felülnézetből az óramutató járásával ellentétesen), úgy hogy a kívül levő, jobbkeze a szemét takarja, valahogy így:\\
\includegraphics[scale=0.65]{picard-facepalm.jpg}\\
a bal kezét a jobb könyöke alatt kiteszi. Megy körbe, valaki rácsap a kint levő kezére és ki kell találnia, hogy ki volt az. Ahhoz, hogy szembeforduljon az emberekkel, csak a kör közebe felé fordulhat (azaz neki balra). Ha eltalálta, akkor cserélnek, ha nem, akkor megy tovább.

\item \textbf{"Szivi ha szeretsz, mosolyogj rám":} Egy ember van a kör közepén, odamegy valakihez és ezt mondja:\\
``Szivi ha szeretsz, mosolyogj rám!''\\
Ha az illető elmosolyodik / elneveti magát, akkor helyet cserélnek; ha nem akkor megy tovább. A játék elején meg kell állapodni, hogy mi számít nevetésnek.

\item \textbf{Angol bulldog 1,2,3:} Egy sáv van a pálya közepén, abban áll egy ember, a többieknek át kell futniuk a sávon. Ha valakit elkap a középső, felemeli és kimondja, hogy ``Angol bulldog 1,2,3'' és eközben nem ejti le, akkor az az ember is bent marad a sávban. Előbb-utóbb majdnem mindenki bent lesz. Aki utolsó volt, az nyer.

\item \textbf{Kanapé:} Két csapatban játsszuk. Körben székek vannak, eggyel több, mint ahány ember van. Három egymás melletti széket kijelölünk, ez lesz a kanapé (ha van egy háromszemélyes kanapé, az még jobb). A játék célja, hogy a kanapé mindhárom helyén egy csapat emberei legyenek, ekkor van vége a játéknak. Két csapatra osztjuk az embereket és leültetjük őket felváltva úgy, hogy a kanapé két szélén az egyik csapatból való játékos, középen másik csapatból való és az üres hely a kanapé jobb szélén ülő mellett jobbra legyen. Minden játékos leírja egy kis papírra a nevét, összehajtja és bedobjuk egy kalapba. Mindenki húz egy nevet. Az a játékos, akitől jobbra van az üres hely mond egy nevet. Akinek a papírján ez a név szerepel az odaül az üres helyre és kicserélik a papírjaikat. Tehát a játék elején a kanapé jobb oldalán ülő hív, ha Katát hívja, akkor akinél a Kata név van (nem az, aki a valódi Kata a csoportban), odaül az üres helyre és a Kata cetlit átadja a hívónak, ekkor a hívó lesz Kata és a korábbi Kata pedig a hívó papírján levő név. A megüresedett széktől balra ülő játékos (tehát akinek jobbra van az üres hely hív egy nevet). A példánál maradva, ha tudja, hogy a kanapé jobb oldalán ülő embert el akarja hozni onnan, akkor Katát mond; ha viszont egy csapatban vannak, akkor semmiképp nem mond Katát. a saját papírunkon levő nevet nem lehet hívni és semmilyen módon nem lehet kommunikálni a csapattagok között; még annyit sem, hogy ``Tudod, hogy kit kell hívni!'' vagy ``Ugye emlékszel melyik név van Sanyinál?'' Ha a kanapén három ugyanabból a csapatból való ember ül, akkor az a csapat nyert.

\item \textbf{Cápás:} Kirakunk a terembe újságpapírlapokat és elindítunk egy zenét vagy a játékvezető azt mondja, hogy indulhat a játék. A játékosoknak az újságpapírok között kell járkálniuk (tengerben úszkálnak). Ha megáll a zene vagy jelez a játékvezető, akkor mindenkinek fel kell állnia egy újságpapírra (ki kell menekülni a szigetre, mert jön a cápa). A játékvezető vár pár másodpercet, majd körbejár és ha valakinek a padlóhoz ér a cipője vagy bármilyen testrésze, akkor az kiesett. A játékvezető kivesz pár újságpapírt és jöhet a következő kör. A játékosok a szigeteken egymást a hátukra, nyakukba vehetik, de ha a tartó ember lelép, akkor a hátán/nyakában levő is kiesik. Ahogy fogynak az újságpapírok egyre kevesebb ember fér fel és folymatosan esnek ki. Az az egy/pár ember nyer, akik a végén bentmaradnak.

\item \textbf{Egymás hátára rajzolós:} Van két vagy több csapat, csapatonként sorbaállunk (egymás hátát nézzük), mindenki kap egy papírt és csapatonként egy-egy filc megy majd végig. A játék elején a vezető a hátul állók hátára tesz egy papírt és rajzol rá valami egyszerűt, de minden csapatnak ugyanazt (pl egy virág vagy egy ház). Ezután akin rajzolt az rajzol a papírjára úgy hogy az előtte állónak a hátára teszi; azt rajzolja le, amit érzett, hogy rá rajzoltak. Végül az első is lerajzolja amit érzett és akié a legjobban hasonlít az eredeti elképzelésre, az nyert. A másik csapatról nem szabad lesni.

\item \textbf{Állathangos párosítós:} Páros létszám szükséges, állatneveket osztunk ki emberneknek, egy fajtát két embernek, mindenki csak a sajátját tudja. A játékvezető jelzésére indul a játék, ekkor mindenki elkezdi a saját állatjának a hangját adni és meg kell találniuk a párjukat. Lehet egy körös játék is, de lehet azt is, hogy az utolsó pár ember kiesik és új köröket játszunk, míg végül csak a nyertesek maradnak.

\item \textbf{Húzd át a királyt a túloldalra:} Két csapat, 3 részre osztott pálya, egy középső nagy rész és két oldalsó kisebb sáv. Mindkét csapat választ egy királyt maguktól, de ezt csak ők tudják. A cél, hogy játékindítás után a saját szélső sávunkba behúzzuk, bevigyük a másik csapat királyát. Ezt úgy tehetjük meg, hogy odamegyünk páran valakihez és megpróbáljuk behúzni a saját sávunkba, közben persze ők is visznek minket. Ha a királyt sikerült behúznunk, akkor a másik csapat bemondja, hogy mi nyertünk és vége a játéknak; ha nem a király volt az, akkor keressük tovább. A királyt lehet védeni de nem jó a feltűnés, mert akkor tudják kit kell bevinni. Vigyázzunk egymás testi épségére.

\item \textbf{Kézre X:} Egész alkalom alatt játszható játék. Van egy vadász, akit a játékvezető jelöl ki és senki nem tudja, hogy ki ő. A vadász az emberek kézfejére X-et rajzol egy filctollal. Akinek rajzolt az kiesett. Feltűnés nélkül kell az X-eket rajzolnia. Közben a játékban levők tippelhetnek a játékvezetőnél, hogy ki a vadász (mert például látták őket kimenni az ajtón és az egyikőjük X-szel a kezén jött vissza). Ha valaki kitalálja, hogy ki a vadász, akkor nyertek a prédák. Ha rossz a tipp, akkor kiesik az is, aki tippelt, meg az is, akire tippelt. Ha mindenkinek van X-e, akkor nyert a vadász.

\item \textbf{Számháború:} A játékosokat két csapatba osztják, amiket legtöbbször kék és piros színnel jelölnek meg. Minden játékos kap egy fejpántot vagy papírdarabot, amin egy négyjegyű szám van; ezt a homlokára kell kötnie. A játékosnak a pályán bujkálva meg kell támadnia az ellenséges csapat játékosait, mégpedig úgy, hogy hangosan kimondja az ellenség számát. Ha eltalálta, akkor az ellenséget "leolvasta". A leolvasott játékos kiesik. A játék célja az ellenséges csapat minden játékosának kiolvasása.\\
\url{https://hu.wikipedia.org/wiki/Sz%C3%A1mh%C3%A1bor%C3%BA}

\item \textbf{Tölts és lőj:} Körben térdelnek vagy ülnek a játékosok. 2-8 emberrel érdemes játszani, több játékosnál már nehéz követni a játék menetét. A játékot ütemre kell játszani, 4 ütem egy akció. Mindenkinek van egy fegyvere, amit fel tud tölteni és rálőni másra. Háromféle mozdulat van: Töltés, védekezés és lövés. Egyszerre csinálja az összes ember az ütemet ezért egy akcióban egy játékos vagy tölt vagy védekezik vagy lő. Egy akció négy ütemből áll, ebből az első kettőben a két tenyerünkkel a combunkra ütünk egyet-egyet, a második kettőben az mozdulatnak megfelelően teszünk. Ha töltünk, akkor a hüvelykujjainkkal a vállunk mögött hátramutatunk (kétszer, mert így jön ki az egy akció). Ha védekezünk, akkor keresztbetett kézzel rácsapunk a vállainkra (jobb kéz bal váll, bal kéz jobb váll). Ha lövünk, akkor mindkét kezünkkel piszolyt mutatva rálövünk valakire (hiába van két kezünk, egy akcióban csak egy emberre tüzelhetünk). Ahhoz hogy valaki tudjon lőni, töltve kell, hogy legyen a fegyvere (ezért érdemes az első körben mindig tölteni, mert addig úgysem lőhetnek ránk). Ha van töltény a fegyverünkben, akkor lőhetünk. Ha valakire rűlövünk és ő épp védekezik, akkor lepattan róla a golyó és nem esett ki; ha épp tölt vagy lő, akkor kiesett; ha egymásra lövünk egyszerre, akkor mindketten kiesünk. Ha valaki nem veszi észre, hogy kiesett, akkor szóljunk neki; több ember esetén nem biztos, hogy észreveszi. Ha valaki látványosan elrontja az ütemet az is kiesik. Van olyan változat, amiben ha 5 töltényt beletöltünk a fegyverünkbe, akkor az átviszi a páncélt.

\item \textbf{Kanalas:} Szükséges egy pakli franciakártya vagy UNO. Körben ülünk mindenkinél van 4 lap, az elsőnél a pakli maradéka letéve, ő felhúz egy lapot és az így nála levő 5ből lerak egyet, ezt felhúzza a következő és ő is lerak egyet, közben az első már újat húzott. Így mennek körbe a lapok, az utolsó maga mellé pakolja az általa letett lapokat. Ha valakinél összesjön 4 darab ugyanolyan (mondjuk 4db  5-ös), ekkor a középen levő kanalakból az felkap egyet, mire mindenki megpróbál felkapni egyet. Középen eggyel kevesebb kanál van mint ahányan játszanak, így egy embernek nem jut, ő kiesik és vele együtt egy kanál is. Az elsőnek kanalat fogó játékosnak be kell mutatnia, hogy tényleg megvolt a négy egyforma.

\item \textbf{Jelenet:} 4-6 önként jelentkező szükséges. Egy bent marad a teremben, a többiek kimennek, a bentinek elmondunk egy sztorit, behívunk egy embert, aki megfigyeli, ahogy a bent lévő elmutogatja a jelenetet, ő a következő behívottnak elmondja, hogy mit látott, ő megint mutogat, következő beszél ... az utolsó elmondja, hogy mi lehetett az eredeti sztori. Nem verseny, inkább csak nevetni lehet rajta. Lehet úgy is játszani, hogy az eredeti sztoriban meghatárounk 8-10 szót és megszámoljuk, hogy az utolsóban mennyi szerepelt ezekből.

\item \textbf{Egy mondat körbesúgása:} Körben állunk/ülünk. A legelső/járékvezető belesúg a mellette ülő fülébe egy mondatot, az továbbsúgja a mellette ülőnek, a végén jót nevetünk, hogy mi jött ki belőle. Nem lehet visszakérdezni, mindenki csak egyszer mondhatja tovább az általa hallott mondatot.

\item \textbf{Aláírásgyűjtés:} A játék előtt (pár nappal, héttel) kell mindenkitől kérni egy dolgot, ami vele történt, ő csinált, rá igaz és különleges; de a többiek nem tudják. Ezeket összesgyűjteni és kinyomtatni lapokra úgy, hogy mindegyik lapon szerepel az összes és mellette egy kis hely. A játék során mindenki kap egy ilyen lapot és egy tollat és odamennek egymáshoz és megkérdezik, hogy pl.: Te vagy aki már vezetett traktort?; ha igen akkor aláírja, ha nem akkor megy tovább és mástól kérdez, egy embertől egyszerre egyet lehet kérdezni, utána tovább kell állni. Lehet valami többekre igaz, akkor előfordulhat, hogy valaki lapján X, valakién pedig Y írta alá ezt a rubrikát. Ha valaki hiányzik, akinek az állítása nagyon specifikus és biztos nem igaz másra, akkor azt a játék elején húzzuk ki a papírokról. Aki először összeszedi az összes állítást, az nyer.

\item \textbf{Rakd ki a szót:} Több csapatban játsszuk, mindegyik kap 10-15 betűt (amilyen betűket kap egy csapat, olyanokat kap a másik is), a játékvezető körülír egy szót és ki kell azt találni, majd kirakni, amelyik csapat a leggyorsabban kirakja az kap pontot, jöhet a következő szó. A játékezető csak olyan szavakat írjon körül, amik kirakhatók az adott betükből. Ha valamelyik betűből több kell egy szóhoz, akkor abból kapjon minden csapat legalább annyit, amennyi kell.

\item \textbf{Találd ki ki írta:} Mindenki kap egy papírt, tollat és 5 percet, hogy írjon 4 szót, kifejezést, ami hozzá kapcsolódik, de nem egyértelmű (valamint a nevét, hogy tudja a játékvezető), ezután összeszedjük a papírokat, a játékvezető felolvassa az első papírt és ki kell találni ki írhatta, aztán a követezőt; a nevetés a lényeg, nem a pontozunk.

\item \textbf{Értelmező szótár:} Hozunk egy értelmező szótárat vagy régi szavak gyűjteményét, keresünk olyan szavakat, amit nem ismernek az emberek, felolvasunk egyet és mindenki a kapott cetlijére leírja, hogy szerinte mi lehet az, a játékvezető leírja a helyeset is egy papírra, ezután összeszedjük a papírokat és felolvassuk mindet, közben az emberek tippelhetnek, hogy szerintük melyik az igaz (egy ember egyre szavazhat csak). Aki eltalálta hogy melyik a jó (ez csak akkor derül ki, ha mindet felolvastuk), az kap két pontot. Akiére tippeltek az kap annyi pontot ahányan tippeltek rá. Lehet több szóval is (5-6) játszani, ekkor csak azutén szedjük összes a papírokat, miután mindengyik szóról írtak. 

\item \textbf{Csokivadászat:} A kör közepén 5-6 különböző csoki és/vagy édesség. Megy körbe néhány dobókockapár, ha valaki duplát dob és tudja az egyik kaja nevét, akkor azt magához veheti, ha elfogyott középről vagy azt akarom ami másnál van, akkor azt úgy tudom megszerezni, hogy ha tudom a kaja nevét és annak a teljes nevét, akinél van; ha nem tudom senkijét/semmiét akkor nem szerzek semmit; egy ember több nyalánkságot is összeszedhet.

\item \textbf{Néma sor:} A játék elején a játékvezető mindenkinek ad egy számot, amit csak az illető tud (n ember esetén 1-től n-ig osztjuk ki a számokat). A játék célja, hogy sorbaálljanak az emberek növekvő (csökkenő) sorrendben úgy, hogy nem lehet közben beszélni és mutatni sem hogy hányas a tiéd. A kreativitásuk oldja meg, hogy hogyan fejezik ki, hogy ők hányas szám.

\item \textbf{Faggass az ABC-vel:} Mindenki kap egy papírt, amin az ABC betűi szerepelnek. (A különleges betűk nem kellenek:/DZS, Q, X, Y, W, ... és a duplákat összes lehet vonni: i-í, o-ó, ö-ő, ... ). A játék során körbe kelle menni kérdezgetni embereket és megtudni róluk dolgokat, amiket a kezdőbetűk segítségével egy-egy betű mellé fel tudsz írni. Egyszerre csak egyet kérdezhetsz egy embertől. A választ egy mondatban felírhatom a megfelelő betű mellé. Ha kérdeztem valakitől, akkor tovább kell mennem máshoz, nem kérdezhetek többet tőle. Aki először ír minden betűhöz, az nyer. A kérdések specifikusak kell, hogy legyenek Pl.: ``Milyen háziállatod van?'' - ``Kutya'' - És akkor felírhatom, hogy XY-nak van egy kutyája, a ``K'' betűhöz. Ha Sanyihoz megyek oda, akkor nem írhatom fel az S betűhöz a válaszát, például ``Sanyinak a fekete a kedvenc színe'', de az F betűhöz igen. Nem lehet azt kérdezni, hogy ``Mondasz egy \emph{Z} betűs dolgot magadról?''. Ha olyan választ kapunk, amit nem tudunk beírni, akkor továbbállunk.

\item \textbf{Név Roulette:} Két egymást kívülről érintő körben kell felállni, mindenki a köre közepe felé fordul. Az érintés pontot meg kell jelölni valamivel (pohár, kislabda, szék, akármivel). Elindul a zene, elkezdenek a körök forogni. Amikor a zene megáll, akkor akik az érintési pontnál vannak megfordulnak és aki előbb mondja a másik nevét, az nyer, a vesztes átmegy a másik körbe. Addig tart, amíg az egyik kör el nem fogy, vagy le nem fújják a játékot.

\item \textbf{5 perc alatt 30 másodperces reklám:} 5 percet kapnak a csapatok, hogy egy megadott dologhoz reklámot készítsenek. Kell készülni előre tulajdonságokkal, amik az eladandó állatra/dologra jellemzőek és ezt a csapatoknak odaadni/kivetíteni. Ezt a reklámot elő is kell adni.

\item \textbf{Evőversenyek:} Zabkása kézzel, erőspaprika, bébikaja, marshmallow, háztartási keksz, stb... Több csapatban is lehet játszani, ekkor a csapatok delegálnak egy-egy jelöltet.

\item \textbf{Babakaja roulette:} Körben ülünk és egy labda/bomba/akármilyen tárgy megy körbe. Akkor adhatod tovább, ha helyesen tudtál válaszolni a kérdésre, ha nem, akkor új kérdést kapsz. Akinél felrobban a bomba (Tik-tak-bumm játékból ki lehet venni), vagy lejár az idő, az kell, hogy egy kanál bébiételt megegyen. (vagy akármit akármekkora előre megszabott mennyiségben). Sok kérdéssel kell készülni, amik többnyire kitalálhatóak/tudhatóak.

\item \textbf{Kulcsszavak:} A játékvezető készül egy előre leírt szöveggel (10-15 mondat). Kiválasztunk 4 embert a társaságból és egy kivételével mindenkit kiküldünk. Az első embernek felolvassa a játékvezető a szöveget, majd az elmeséli a másodiknak, az a harmadiknak és az a negyediknek. A negyedik pedig hangosan elmondja a történetet. Az elején meg kell jelölni 10 szót a sztoriban és ahány azok közül elhangzik az utolsótól, annyi pontot kapnak. Az első játékos sem ismerheti ezt a 10 szót. (Mindig csak az jön be, aki hallgatni fogja a sztorit és mindenki csak egyszer hallhatja)

\item \textbf{Jelszó:} Két-két játékos versenyez egymás ellen. \emph{Kitaláló-1} és \emph{Szómondó-1} vannak együtt, az ellenfél pedig \emph{Kitaláló-2} és \emph{Szómondó-2}. Kivetít a játékvezető egy szót úgy, hogy az csak a két \emph{Szómondó} láthassa (például: \emph{villáskulcs}). Az egyik páros kezd: \emph{Szómondó-1} mond egy szót, ami nem lehet a kivetített szó és nem is lehet annak része vagy átalakított formája (pl: \emph{villa, ás, kulcs, villám, kulacs}), de a célja, hogy a társa ez alapján kitalálja a kivetített szót. Például azt mondja, hogy \emph{csavarhúzó}. Erre a társa, \emph{Kitaláló-1} tippel egy szót (pl: \emph{szerszám}). Minden játékos csak egy-egy szót mondhat, nem kommunikálhat másképpen. Ha kitalálta, akkor nyertek egy pontot, ha nem, akkor jön a másik csapat. \emph{Szómondó-2} mond egy szót (pl: \emph{kanál}). \emph{Kitaláló-2} tippel egyet. Neki annyival könnyeb dolga van, hogy hallott már két szót (\emph{csavarhúzó} és \emph{kanál}) és hallott egy rossz tippet (\emph{szerszám}). Ha kitalálta, akkor övék a pont, ha nem, akkor újra az első csapat jön. 

\item \textbf{Rabszolgamunka:} Legalább két csapatban játsszuk. Mindegyik csapat kap 10 műanyagpoharat. A csapatok elhelyezkednek egy vonalban, 1-2 méter legyen mindegyik csapattag között. A játékvezető jelzésére az első játékosok felépítenek egy piramist 4-3-2-1 pohárból, majd a második játékossal együtt megpróbálják a második játékos elé eljuttatni a piramist úgy, hogy az ne boruljon le. Ha valamelyik darabja leesik, akkor ott újra kell építeniük  (csak kettejüknek) és továbbvinniük. Ha a második játékos elé ért a piramis, akkor annak le kell bontania, egy kupacba raknia, majd újra felépítenie a piramist. Ezután a harmadik játékossal eljuttatják a harmadik elé és így tovább, amíg el nem jut az utolsó emberig. Amelyik csapat hamarabb végez, az nyer. Nem használhatnak a kezükön kívül mást. Mozgatásnál választhatnak, hogy megemelik a piramist vagy megpróbálják csúsztatva odébb juttatni.
\url{http://www.ultimatecampresource.com/site/camp-activity/slaves-of-job.html}

\item \textbf{Te, Én, Bal, Jobb:} Neves játék. Körben ül mindenki, egy (vagy lehet több emberrel is) ember pedig középen áll. Odamegy valakihez, rámutat és azt mondja, hogy ``én'' vagy ``te'' vagy ``bal'' vagy ``jobb'' és az ülőnek ez alapján kell mondania egy nevet:\\
``én'' esetén a középen álló nevét\\
``te'' esetén a saját nevét\\
``bal'' esetén a tőle jobbra ülő nevét\\
``jobb'' esetén a tőle balra ülő nevét\\
Tehát a középen levő ember felől nézzük az irányokat. Annyi ideje van kimondani ezt, amíg a középen álló hangosan el nem számol 5-ig korrekt sebességgel. Ha nem sikerül, vagy elrontja, akkor helyet cserélnek

\item \textbf{WC-papír bemutatkozás:} Körben ülünk, mindenkihez odamegy a játékvezető és megkérdezi hány WC-papír darabot szeretne. Letépi neki a kívánt számút, majd amikor mindenki megkapta amennyit szeretett volna, akkor mindenkinek ahány darab van nála, annyi féle dolgot kell magáról mondania. Nem szabad előre elárulni, hogy ezért kell a WC-papír. Lehet csavarni a dolgon azzal, hogy azt kérjük, annyi kockát tépjenek, amennyire szükségük lenne egy 3 napos sátortáborhoz vagy egy kávés uborkás babfőzelék után.

\item \textbf{Forgalmi dugó:} Felsorakozunk egy sorba, a sor közepén van egy üres hely és mindenki az üres hely felé (középre) fordul. Mindenki mellé és az üres hely mellé is lehet tenni egy széket, így tudjuk pontosan, hogy hol vannak a helyek. A feladat az, hogy a két oldal helyet cseréljen. A szabályok: nem lehet hátrafelé haladni; csak üres helyre mozoghatsz előre; nem "ugorhatod át" csapattársadat (csapattárs, aki ugyanarra néz); egyszerre csak egy ember mozoghat; egy helyen csak egy ember állhat egyszerre.

\item \textbf{Zombi:} Nagyobb terület kell hozzá. Vannak zombik (emberek, akiknek úgy kell járni mint egy zombinak, üvöltöznek is közben) és nekik a céljuk, hogy megvédjenek egy területet (cél zóna, általában egy 4-8 méter átmérőjű kör). Vannak normál emberek, nekik a céljuk, hogy bejussanak a cél zónába, ekkor szereznek pontot (ha egy ember bejut az egy pont). Az emberek egy start zónából indulnak és be kell jutniuk a cél zónába anélkül, hogy egy zombi megérintené őket; ha ez mégis megtörténik, akkor ott ahol vannak leülnek, leguggolnak. Ha beért valaki a célba, akkor van egy pont és mehet vissza a starthoz újraindulni (ilyenkor nem foghatja a zombi, de neki is valahogy egyértelműen jeleznie kell, hogy visszafelé megy, például feltartott kézzel). Cél, hogy x darab pont meglegyen (létszámtól függ) adott idő alatt. Ezen felül vannak gyógyítók, őket külön meg kell jelölni (pl kendő a csuklóra), nekik a feladatuk, hogy ha egy embert elkaptak a zombik, akkor odamehetnek, kézenfogva visszavihetik a starthoz és akkor az ember újra indulhat. Egy gyógyító egyszerre több embert is vihet. A gyógyító ezenfelül védheti is az embereket a testével, de nem erőszakosan. A gyógyítónak egészen a start zónáig vissza kell vinnie az embereket ahhoz, hogy azok újrainduljanak. A játékvezető számolja a pontokat.

\item \textbf{Elnökös:} Van egy elnök és neki 5-8 testőre /létszámtól függően/, akiket magának választ. Vannak a bérgyilkosok (mindenki más de legalább 3szor annyi, mint a testőrök), akiknek az a feladata, hogy megöljék az elnököt, mielőtt az megtalálja a kincset. A játék elején a bérgyilkosok eldugják a kincset a pályán, hogy az elnök és az emberei ne tudják. Az elnöknek van egy háza, onnan indulnak a testőrök és az elnök. A bérgyilkosok nem mehetnek be a házba. A játék indítása után a kincshez nem nyúlhat már senki, csak az elnök. A testőrök ha megérintenek egy bérgyilkost, az kiesik; ugyanez visszafelé, de a testőröket csak úgy lehet kiejteni, ha hátulról fogod meg (pl a hátát vagy a vádliját). Ha az elnököt megérintik a bérgyilkosok akárhol, akkor vége a játéknak, az elnök nem ejthet ki bérgyilkost. Célszerű úgy játszani, hogy a testőrök először megkeresik a kincset és utána hozzák ki az elnököt körbefogva, hogy szerezze meg a kincset.


\item \textbf{10 másodperces bújócska:} Több csapatban lehet játszani. Az egyik csapatból valaki kimegy a játékterületről. 30 másodperc van, hogy mindenki elbújjon, ekkor bejön a hunyó és 20 másodperce van, hogy megtaláljon embereket. Akit meglát és kimondja a nevét az megvan. Ha saját csapattársát látja meg, annak is ki kell mondani a nevét. Nem beszélhetnek előre össze a csapattagok a saját csapatukból való hunyóval, hogy ki hova fog bújni. 20 másodperc után összeszámoljuk, hogy melyik csapatból hány embert talált meg a hunyó. Ezután másik csapatból való hunyó megy ki. 













\item \textbf{Pohárlefújós:} / Egy asztalt le kell fedni eldobható műanyag poharakkal, majd egy lufit kap minden csapat. Amelyik csapat előbb lefújja a lufiba fújt levegővel az összes poharat, az nyer.
\item \textbf{Mentsd a széket!:} / Rendezzük a székeket egy nagy körbe. Valaki álljon a kör közepén, mindenki más széken üljön és legyen egy extra szék. (20 fő esetén 2, 30 fő esetén 3, stb.)

A játék célja, hogy a középen álló személy leüljön egy üres székre, az összes többi ülő személy pedig próbálja megakadályozni őt ebben. Ezt úgy tehetik, hogy eggyel jobbra vagy balra ülnek, mielőtt a középen álló ember odaül (nem lehet a körön keresztül szaladni például). Egy idő után a középső ember sikeresen leül és a mellette jobbra vagy balra ülő ember (amelyik lassú volt odaülni) áll fel középre.
\item \textbf{Ügyetlen Bowling:} / A játék célja, hogy minél bénább legyél bowlingban, vagyis minél kevesebb pontot érj el. A pályát mi asztalok oldalra fektetésével határoztuk meg, egy bowling pályát alakítva velük. A pálya szélességét úgy állítottuk fel, hogy a piramisba felállított 10 bábu mellett ne férjen el a labda. A játék célja, hogy minél kevesebb pontot gyűjtsünk, azaz minél kevesebb bábut döntsünk le. Ha valaki az összes bábut lelöki a gurításával, akkor 0 pontot kap érte (a legkevesebb elérhető pontszám).

\item \textbf{Hernyó verseny:} / A csapatok felsorakoznak a kijelölt vonal mögött. Minden csapat választ magának egy egy szótagú hívószót (vagy a csapatnévvel is játszható ez), majd mindenki megfogja az előtte levő vállát. Csak páros lábú szökkenéssel lehet előre haladni, de egyszerre csak egy ember ugorhat. Ez a gyakorlatban úgy néz ki, hogy az indításkor az első ember előre szökken (közben fogja az ő vállát a mögötte álló ember), majd utána a második, majd a harmadik és így tovább, amíg el nem érik az utolsó embert. Amikor az utolsó ember is szökkent, akkor előrekiáltja a hívószavukat és ezt hallva az első szökkenhet egyet ismét. Szóval mindenki meg kell, hogy várja az előtte levő szökkenését, hogy ő is haladhasson, az elsőnek pedig az utolsó jelez, amikor ő is szökkent már. Az a csapat nyer, akitől az utolsó ember először áthalad a kijelölt célvonalon.

\item \textbf{Ejtsd le a fedőt:} / Mindenki egy körben ül székeken, egy ember pedig középen áll egy edény fedőjével a kezében (mi kanállal játszottuk, az is bőven megfelelt a célnak; lényeg, hogy hangos legyen amikor leesik). Odamegy egy ellenkező neműhöz a körben és felhúzza, majd fogja az ő kezét. Akkor a felhúzott ember keres egy ellenkező neműt és húzza fel. Ez így folytatódik és hosszabbodik a sor, míg az első ember el nem ejti a fedőt. Amikor az felszólal, akkor mindenki aki állt megpróbál leülni. Aki állva marad, az felveszi a fedőt és új kör kezdődik.

\item \textbf{Beltéri röpi:} / A háló egy sor asztalból áll a termen keresztbe, amin székek vannak. A székek felváltva az egyik és a másik csapat felé néznek. A játékot strandlabdával kell játszani. Minden csapatnak 3 érintése van, de fejjel és lábbal akárhányszor beleérhetnek (mi lábbal sem engedtük a lámpákat megóvandó). A falak és minden teremben levő eszköz is a játéktér része, megpattanhat rajtuk a labda, de ha plafont ér az, akkor a másik csapaté a pont. Pontozás és szervák igény szerint (mi 11-es meccseket játszottunk, ahol 3 szerva után cseréltek szervát).

\item \textbf{Vak vadász:} / Kell egy önkéntes, aki a vadász lesz. Az ő szemét be kell kötni és kap egy medence nudlit (mi korábban már más játékhoz kettévágtunk jópárat, úgyhogy így nem túl hosszú darabbal tudtunk most játszani). A vadász feladata az, hogy megüsse a nudlival a többi játékost, akit eltalál az kiesik és lejön a pályáról. Az a játékos nyer, aki az utolsó életben maradva. Ki kell jelölni a játékterületet, hogy a vadász feladata ennyivel egyszerűbb legyen. Ahogy fogytak a játékosok, úgy nyomtuk össze a területet. A vadász 10-15 másodpercenként elkiálthatja magát, hogy “Marco” és akkor minden életben levő játékosnak vissza kell kiáltania, hogy “Polo”, így a vadász nem csak a sötétben tapogatózik.

\item \textbf{Egyszerűsített rögbi:} Egyszerűsített rögbi szabályzat\\
A játék célja: A játék során két csapat versenyez egymás ellen. Céljuk, hogy minél több pontot szerezzenek. Az nyer, akinek a megszabott idő végén több pontja van. Ha egyenlő az állás, akkor hosszabbítás következik, ha a hosszabbítás végén is egyenlő az állás, akkor aki a következő pontot szerzi az nyer.\\
\\
A pálya: (TODO rajz) téglalap alakú. 5 részből áll:
B csapat célzónája (A csapat térfele), B csapat dobózónája (A csapat térfele), Senki földje, A csapat dobózónája (B csapat térfele), A csapat célzónája (B csapat térfele).\\
Célzónák: keskeny sávok a pálya rövidebbik végein. Max (4méter; pálya hosszának 1/20 része).\\
Dobózónák (pálya 1/5 része)\\
\\
Törekedni kell az egyenlő feltételeket biztosító térfelek kialakítására. Ha ez nem sikerül, akkor az aránytalanságot a lehető legnagyobb mértékben csökkenteni kell. Amennyiben bármelyik csapat igényli az aránytalanság miatt (ennek indokoltnak kell lennie), abban az esetben a játszma félidejében térfelet kell cserélni a csapatoknak. Aránytalanság például a pálya ferdesége, az egyik csapatnak szemébe sütő nap, egyenetlen talaj.
A pálya széleit és a zónák határvonalait egyértelműen jelölni kell. A vonal a pálya része. A zónák határai a pálya közepe felé eső zónához tartoznak (senki földje és dobózóna határa a senki földjéhez; adott csapat térfelén lévő dobózóna és célzóna határa, az adott térfél dobózónájához tartozik.\\
\\
Fogalmak:\\
Teljes tartás: Ha a játékosnál van a labda olyan állapotban, hogy ezzel az elhelyezkedéssel (póz, beállás) tartós ideig (min. 6 mp) képes lenne tartani a labdát, akkor az teljes tartás. Teljes tartásról csak a pálya határain belül a talajjal érintkező játékos esetén beszélünk, ha a labda eközben nem érinti a talajt és a pálya légterében van.\\
Érintés: A labdajáték közben, ha egy játékos bármely testrészével hozzáér a labdához, akkor érintésről beszélünk, kivéve, ha a labda teljes tartásával együtt történik az érintés (mert az már teljes tartásnak számít). Pl ha egy játékos egy vagy két kézzel lecsapja reptében a labdát, akkor az érintés, ha viszont egy vagy két kézzel úgy fogja meg, hogy képes lenne tartósabb ideig azt tartani, akkor már teljes tartásról beszélünk. TODO (mi van a rúgással, fejeléssel?)
Érintés során a játékosnak a pálya határain belül a talajjal kell érintkeznie, vagy a levegőben lennie, úgy hogy az elrugaszkodáskor a pálya határain belül érintette a talajt.\\
Passz: az egyik csapat sikeres labdaátjátszása a saját csapattársukhoz. A labdát dobni kell, nem lehet kézből kézbe átadni. A labdának a pálya légterében kell maradnia. A labdajáték teljes tartással kezdődik és azzal is végződik. Ha A csapat dobásába B csapat beleér (érintés), de A csapat tagja elkapja (lehet az is, aki dobta), akkor folytatódik a játék. Passznak minősül ha nem ugyanaz volt a dobó és az elkapó, nem minősül passznak, ha ugyanaz volt a dobó és az elkapó. A csapat dobásába akármennyi játékos beleérhet (érintés), akár A, akár B csapatból, de amint a labda teljes tartásba kerül az számít csak passznak. (pl senki földjéről érintéssel nem lehet bejuttatni a cél zónába, kell a dobó zónában minimum egy teljes tartás). A passz során ha a másik csapat teljes tartással megszerzi a labdát vagy a labda érinti a talajt vagy elhagyja a pálya légterét (TODO rúgás), akkor a másik csapat következik. Az A csapatnak az A csapat célzónájába csak az A csapat dobózónájából lehet passzal bejuttatni a labdát, ellenkező esetben a másik csapat jön.\\
Érvénytelen passz esetén a másik csapat jön onnan, (ahol megszerezték a labdát\footnote{Ha az A csapat dobózónájában vagy célzónájában A csapat labdát szerez (pl. B csapat elejti), akkor A csapatnak ki kell hoznia a labdát a Senki földje és A csapat dobózónája határáig minimum (senki földje felé lehet kijjebb vinni) és innen jön az A csapat. Ezt a kihozást minél gyorsabban kell elvégezni és a másik csapat nem akadályozhat ebben (ez nem a játék része, tehát bárhogyan oda lehet juttatni). Ha ezt elmulasztotta a csapat és akárki észreveszi, akkor végre kell hajtani a labda visszavitelét. (nem kerül a másik csapathoz a labda). Ha nem veszik észre, hogy vissza kellett volna vinni a labdát, de fél percen belül jelzi ezt valaki, akkor el kell végezni a labda visszavitelét és az addig szerzett pontok elvesznek.}) vagy (ahol a másik csapat elejtette a labdát\footnotemark[\value{footnote}]) vagy (ahol kiment a pályáról a labda/labdás játékos a másik csapatból\footnotemark[\value{footnote}]) vagy (ahol a másik csapat érvénytelenül labdát szerzett /pl senki földjéről A csapat A cél zónájába dobta, ekkor az A cél zóna és az A dobó zóna határától jön a B csapat) vagy (ahol a visszapassz történt, a visszapasszoló helye számít\footnotemark[\value{footnote}]) \\
Labda: a labda rögbilabda, nagysága a korosztályhoz igazítva\\
\\
A játék menete:\\
A játék bírói feldobással kezdődik a pálya közepéről. Az idő a feldobás pillanatában indul. Amelyik csapat teljes tartásba tudja vinni a labdát, ők egyből passzolhatnak, olyan mintha ők kezdenének onnan, ahol a teljes tartást megszerezték. (ha leesik a labda a feldobás után, akkor aki felveszi, azé lesz; ha érintés után esik le, akkor is felveheti akárki). A labdát passzokkal juttatják a játékosok a célzóna felé.
Játékidőben a játék kezdete előtt meg kell állapodni, célszerű 12-15 perces meneteket játszani, szükség esetén 5 perc hosszabbítással.
Érvényes pont szerzése, ha az egyik csapat a saját dobózónájukból teljes tartásból a saját célzónájukba passzolják a labdát és ott teljes tartásban elkapja a csapattárs, akinek mindkét lába a célzónában van vagy oda érkezik. Pontszerzés után a másik csapat jön a saját térfelükön lévő dobózóna és célzóna határáról. (ezek a másik csapat zónái)
A labdával sétálni nem lehet. Ha valaki mégis megteszi, abban az esetben a másik csapat jön, kivéve ha:
-az ellépés elkerülhetetlen volt, például levegőben való labdaelkapás esetén, ha a játékosnak nagy volt a lendülete. Ebben az esetben amilyen gyorsan csak lehet vissza kell lépnie a kiinduló helyére (ahol földet ért) és folytathatja a játékot. A játékosok a labdát birtokló játékostól minimum 40 cm távolságra kell, hogy álljanak, ezzel biztosítva a labda passzolásának lehetőségét. Ha ennek a megsértése miatt hiúsul meg a passz, akkor azt meg kell ismételni.
A labdaszerzés pillanatától számítva 5 másodperc áll a játékos rendelkezésére, hogy passzoljon, ha ezt nem teszi, akkor a másik csapat jön (kezdő játékosok vagy kicsik esetén ezt lehet lazítani vagy elengedni).
Visszapasazolni nem lehet. Tehát, ha \#1 játékos \#2-nek passzol, ő nem passzolhatja vissza \#1 játékosnak. Ha mégis ez történik, akkor a másik csapat jön.
Itt a visszapassz nem lehet érintés sem, tehát \#2 játékos passzába más játékosnak is bele kell érnie, mielőtt \#1 beleér és \#1 játékos nem lehet a passz megszerzője.
Ha a labdát két ellenkező csapatbeli egyszerre fogja meg (teljes tartás), akkor a bírónak fel kell dobnia a labdát, ha az eset a senki földjén történt. Ha az eset bármelyik dobózónában vagy célzónában történik, akkor a passzt adó csapat jön\footnotemark[\value{footnote}]. (ki kell venniük a senki földjéig) todo:ez jó így???
A játékidő leteltekor ha a levegőben van a labda, akkor az első bármilyen érintéssel vagy státuszváltással (teljes tartás, érintés, leesik a labda, kimegy a pályáról) ér véget a játék. Ha ez teljes tartás a cél zónában, akkor az értelemszerűen pontot ér.
Egyéb esetben az idő leteltével befejeződik a menet.

A csapatoktól sportszerű játékot várunk el és jó szórakozás kívánunk \Smiley

\end{itemize}

Széken ülő kosárlabda.\\
Parti-e?\\
Elnök titkár jegyző\\
Tenyér a tenyérhez párbaj, nem lehet ellépni\\
Körben labdadobálás, ne kapja el a középső\\
egymásnak forgatott asztalok, labdával kiütős.\\
Videóból kérdés.\\
Két labda körbe érje utol az egyik a másikat\\
Told ki a körből\\


\end{document}